% !TeX root = ../proposal.tex

% 5000字应该大概有10页

\chapter{背景}

\section{宇宙起源之谜与未知的高能物理}

翻开任意一本哲学史,开篇往往记录着人类文明对于智慧的求索起源于一个著名的问题:“世界的本源是什么”。第一批回答者着眼于日常所见的具体元素:水、火、土、气,虽然已经是对生活经验的初步概括,但仍不能使人信服。后来的哲学家用一些更抽象的概念来回答:“数”、“原子”,虽仍然不是最恰当的答案,却启发了后来的人们。
近代自然科学发展后,物理学(physics)取代了形而上学(meta-physics),用一套行之有效的、基于数学符号和实验分析的方法来探索自然世界,将诸如“世界本源”一类的哲学问题转化为可度量、可验证的科学问题。经过物理学几百年的快速发展,今天的“原子”已经是定义良好的概念,并在特定的情境(覆盖日常生活的绝大多数情境)下承担着“组成世界的基本单位”这一角色。

如果我们能不驻足于“完善德谟克利特的猜想”这种带有浪漫色彩的史诗化表述,而是被这一过程自然而然的“副产物”所吸引:当人们对“原子”的概念理解清楚之后,他们很快(在半个世纪内)发现了原子之内还有更基础的组分,即使在这么小的尺度上依然有丰富的现象,其中蕴含着深刻的物理规律,这些知识最终总结为粒子物理的标准模型,标志着人类现今对于微观世界的全部理解。
所以若我们从空间的角度理解“世界的本源是什么”这个问题,即“构成世界的基本组分有哪些”,那么粒子物理标准模型就是时至今日最好的答案。

但一个哲学问题翻译成科学问题时往往有多种解读的方向,“本源”二字除了从空间上解析到极微观外,还可以从时间上追溯到极早期。而令人兴奋的是,这两种解读导向了同一个方向:宇宙极早期的高能物理过程。

一百多年前,哈勃通过观测遥远星系的红移发现宇宙在膨胀。基于膨胀宇宙论做合理的外推,宇宙早期应处于高温与高密的状态,后来的微波背景辐射(Cosmic Microwave Background, CMB)、原初核合成(Big Bang Nucleosynthesis, BBN)等观测证据佐证了这一猜想,九十年代末基于超新星测距发现的“宇宙在加速膨胀”这一结论更是将膨胀的宇宙论确立为现代宇宙学的标准模型。

在这一标准模型中,我们对宇宙的起源仍知之甚少。
从观测的角度,CMB之前光子与等离子态的核子、电子耦合在一起,导致宇宙这“前38万年”的无法通过光子直接探测,只能寄希望于中微子、引力波等新兴的宇宙学探针,或是通过CMB及之后宇宙大尺度结构(Large Scale Structure, LSS)的观测来间接推断。
而理论的角度,高温高密的早期宇宙对应的正是理论物理所未尝理解的极高能标,时空的极度扭曲与物质的量子性质耦合在一起,导致单独的广义相对论(目前所知描述引力的最佳理论)或量子场论(描述其它三种相互作用的有力工具)都失去了描述和预言的能力,而二者的统一又具有相当的难度。

自人们从对“原子”的研究中一步步走来,不仅发现了全新的强/弱相互作用,还在理论上将它们和电磁相互作用统一(成为粒子物理标准模型中的玻色子们)以来,数代理论物理学家花费了大量精力试图将最后一种相互作用——引力统一进来。
% “弦论”即是这些尝试中最有名的一类,但
他们面临的不只是理论构建上的难度,更是实验检验的巨大困难。
这些可能的“统一理论”工作在极高的能标,而地球上所能达到的最高能标依赖于人类建设的大型粒子加速对撞机,目前最大的是在2012年用希格斯粒子(Higgs)的实验探测补齐粒子物理标准模型的最后一块拼图的欧洲大型强子对撞机(Large Hadron Collider, LHC),它所能达到的能标约为$10^4\,{\rm GeV}$。由于对撞机的建设成本极其高昂,而现在的“最高能标”和“统一理论”之间还有巨大的鸿沟,且在临近的几个数量级内缺乏值得花费成本验证的理论预言,目前人们对大型对撞机的资金投入并不乐观。
但在地球之外,存在着诸多天然的“高能物理实验室”,比如超新星爆发、活动星系核、伽马射电暴等极端天体物理过程能标可达$10^6\,{\rm GeV}$以上,它们产生的中微子和宇宙射线等可以成为探测高能标新物理的窗口。
而宇宙极早期的能标理论上更是高达$10^{13}-10^{14}\,{\rm GeV}$,这一时期可能存在哪些物理过程完全在人类现有知识体系之外,但却与“统一理论”的构建和预言相关。
幸运的是,虽然直接探测的手段尚未完善,但早期宇宙仍可以通过传统的宇宙学观测方法,通过后来的物质分布来推断早期的特征,进而印证、限制或推翻极高能标下的物理理论。而宇宙学本身,经过数十年的迅速发展,已经进入了“精确测量”的时代,无疑将为探测原初宇宙的物理过程提供坚实的基础。

于是我们看到,在两千多年后,“原子论”与“宇宙论”的殊途同归构成了“世界的本源是什么”这一问题的现代表述:“宇宙极早期存在哪些物理过程”。对原初宇宙的探测不仅是宇宙学的圣杯之一,也将为高能物理研究提供丰富的现象。正如“打开原子”后,我们遇到了纷繁复杂的微观世界,从原初宇宙的一两个统计特征开始,我们也将触碰到其背后由更深刻的规律支配的全新世界。
% 如果说现在我们对原初宇宙的无知和大量模型猜测就像古希腊人猜想“世界的本源”一样,那么即将到来的现代宇宙学观测将如同一百多年前从原子中发现了电子。

\section{暴胀理论及其观测证据}

我们从人类现在对原初宇宙的认识开始。
上世纪80年代,为了解决磁单极子不存在的问题,物理学家 Alan Guth 等人提出了暴胀理论(Cosmic Inflation),认为宇宙在极早期经历了一段短暂的指数级膨胀,将磁单极子“推”出了视界之外。
暴胀理论同时解决了在CMB中观测到的“视界疑难”和“平坦性疑难”。前者指的是在大爆炸模型里,CMB上相距甚远,不应该有因果联系的两点在同样的平均温度(2.73K)附近以同样规律微小涨落(涨落为$10^{-5}$量级)。后者指根据CMB功率谱拟合标准宇宙学模型得到的曲率几乎为0,说明宇宙的三维空间是平坦的。在暴胀理论中,今天的可观测宇宙由暴胀前一个极小的区域演化而来,因此CMB上各个位置在暴胀前都有因果联系,所以温度非常相近;而平坦性则是因为极小区域的曲率总是近似为平坦的,换句话说,暴胀抹平了空间曲率。

在解决几大疑难的基础上,暴胀理论还给宇宙大尺度结构的起源给出了自然的解释:暴胀期间共动坐标下的视界(Comoving Horizon)逐渐变小,原本视界内的量子涨落按照共动尺度从大到小依次出视界,被固定为经典涨落(这个过程也称为freeze-in),在暴胀结束后重新进入视界,之后在引力的作用下演化,最终形成我们观测到的宇宙大尺度结构。
% 暴胀理论认为宇宙的大尺度结构有原初的起源
% The distribution of curvature fluctuations is pretty uniform, Gaussian, and scale invariant.

虽然还未完全纳入“标准模型”的框架,暴胀理论已经取得一些初步的观测支持,主要和上一段提到的暴胀理论认为宇宙的大尺度结构有原初的起源有关。
其一是原初涨落进入视界后的演化预言了重子声波振荡(BAO)的存在,在CMB和星系光谱巡天中都得到了验证。
其二是暴胀期间哈勃参数的变化很小,而原初涨落的幅度和其出视界时刻的哈勃参数成正比,因此不同尺度的涨落在出视界时刻的幅度也应是近似相同的(不是严格相同,因为暴胀期间哈勃参数有微小的演化),参数化的表示是功率谱的谱指数$n_s$接近于1。普朗克卫星2018年的数据给出了$n_s=0.965\pm0.004$,与暴胀理论的预言是一致的。
最后,简单的慢滚暴胀理论预言原初涨落近似为高斯场,度量非高斯性大小的参数$f_{\rm NL}$受到慢滚参数的压低$f_{\rm NL} \sim \mathcal{O}(\epsilon)$。CMB的观测暂时也没有测到明显的非高斯性,且给出了一定的限制,比如普朗克卫星2018年的数据给出了目前最严格的限制$f_{\rm NL}=-0.9\pm5.1$。
% 最后,简单的慢滚暴胀理论预言原初涨落近似为高斯场,其三点关联函数相比两点关联函数有一个慢滚参数量级的压低$\left\langle \zeta^3 \right\rangle \sim \mathcal{O}(\epsilon) P_\zeta^2$。CMB的观测暂时也没有测到明显的非高斯性,且给出了一定的限制,例如$f_{\rm NL}=-0.9\pm5.1$。
% tells us "Inflatons are weakly coupled".

\section{原初非高斯性}


除了CMB之外,我们通过测量宇宙大尺度结构也可以反推暴胀结束时的原初场的统计性质。
原初场的统计性质和暴胀时期的物理过程有很好的对应,目前研究最多的功率谱(power spectrum)在统计学上可以完整地刻画一个高斯随机场的性质,它对应暴胀期间粒子的自由传播(free propagation);
而原初场相对于一个高斯随机场的偏移——称为“原初非高斯性”(Primordial non-Gaussianity, PNG)——则需要更高阶的统计学量——比如双谱(bispectrum)——对应的则是暴胀期间粒子之间的相互作用。
因此,通过观测原初非高斯性

\section{“宇宙学对撞机”:原初高能物理的理论进展}

无助推宇宙学对撞机自举(Boostless Cosmological Collider Bootstrap, BCCB (四个词拼起来的奇怪翻译))

\section{宇宙学观测现状}

在理论工具取得突破的鼓舞下,部分理论物理学家开始将目光投入在真实的宇宙学观测数据中挖掘“宇宙学对撞机”的信号。(如\cite{cabass2024boss,sohn2024CCCMB})



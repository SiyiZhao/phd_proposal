% !TeX root = ../proposal.tex

\chapter{研究背景}

\section{释题:从“世界的本源”到“原初宇宙的高能物理”}

% 现代科学的起源可以追溯到古希腊哲学,在后者为前者提供的诸多重要问题中,最早可考的一个是“世界的本源是什么”。
翻开任意一本哲学史,开篇通常记录着人类文明对于“智慧”的求索起源于一个著名的问题:“世界的本源是什么”。哲学史本身并未持续给出与时俱进的答案,而是将这一任务交给科学。
% 作为书写人类思想之发轫的哲学史开篇往往是古希腊的一个著名问题:“世界的本源是什么”,
时至今日,现代版本的“原子论”及其进一步发展后的“粒子物理标准模型”,从组份细分的角度来说,就是目前对这个问题最好的答案。
但相应的物理理论本身提示我们这并非终点——“万有引力”尚不在“标准模型”中,将时空纳入这个框架仍具有理论上的困难。
% 更不用说天文学还提出了“暗物质”“暗能量”这些几乎完全未知的能量组份。
% 近五百年现代物理学的发展让我们理解了日常生活尺度上种种现象的基本规律,虽然在这个尺度上,复杂系统仍有众多悬而未决的问题和转换为生产力的无限潜力,但看起来与“世界的本源”问题更接近的是与日常生活相去甚远的极小与极大两个尺度,即“衔尾蛇”的头与尾。

%%% 这段讲宇宙学的简史其实在逻辑上并不必要,这里只是将“世界的本源”问题导向“宇宙的起源”问题。 但是作为宇宙学方向的博士论文,梳理晚近的历史也许是必要的。
从时间的角度切入这个问题则涉及到宇宙观的演变。
% 直到1920年,Shapley和Curtis还在“大辩论”中争论银河系是不是宇宙的全部,而很快Hubble的观测就将人类的宇宙观从银河系扩大到包含诸多星系的广阔空间。
直到1920年,沙普利(Shapley)和柯蒂斯(Curtis)还在“大辩论”中争论银河系是不是宇宙的全部,而很快哈勃(Hubble)的观测就将人类的宇宙观从银河系扩大到包含诸多星系的广阔空间。%将康德的/哲学中“岛宇宙”引入/转译为科学中的星系。
随后的观测中提出的哈勃定律则进一步揭示了宇宙在膨胀,将宇宙观从简单假设的“静态”引向有实证支持的“动态”。%从简单的静态假设引向有实证的动态断言;
% 一百多年前,哈勃通过观测遥远星系的红移发现宇宙在膨胀,将宇宙观从简单的静态假设引向有实证的动态;且对膨胀状态在时间上做反方向的外推,
基于膨胀宇宙论做合理的外推,宇宙在时间上的“起源”也成为一个自然的概念。
% 外推得到的宇宙早期应处于高温高密的状态,后来的微波背景辐射(Cosmic Microwave Background, CMB, 1964)、原初核合成(Big Bang Nucleosynthesis, BBN, 1960s)等观测佐证了这一点,九十年代末基于更精确的超新星测距得出“宇宙在加速膨胀”更是将膨胀的宇宙论确立为现代宇宙学的标准模型。
之后在宇宙微波背景辐射(Cosmic Microwave Background, CMB, 1964)、原初核合成(Big Bang Nucleosynthesis, BBN, 1960s)、宇宙大尺度结构(Large Scale Structure, LSS, 1970s)等观测证据的支持下,“大爆炸宇宙学”逐渐成为现代科学公认的宇宙观,
到了九十年代末基于超新星测距发现“宇宙在加速膨胀”,进而引入暗能量的概念后,现代宇宙学的标准模型已经能至少在零阶近似下准确地描述宇宙的绝大部分历史。
仍然未知的那部分就是电弱相变之前的“原初宇宙”,% 包括暴胀、reheating(热大爆炸)、大统一相变等。本文专注于暴胀部分。
或者说宇宙的“起源”。因为这一时期的能标极高,尚不在物理学知识体系之内。
% 却也把“宇宙诞生之初”这个“关乎起源”的概念置于现代物理学知识体系之外——这段特殊的历史时期应该处于极高的能标,可以用简洁优美的统一理论描述,但我们在探索这个“统一理论”的过程中却遇到了极大的困难。

% 基于科学史上的经验,
理论物理学家大都认为,目前经过实验认证的最深刻的两大理论——广义相对论(目前所知描述引力的最佳理论)和量子场论(描述其它三种相互作用的有力工具)只是某种更高能标的“统一理论”的低能近似,后者应能同时自然地描述全部四种相互作用。
% 更高能标理论的低能近似,而更高能标的理论应能将四种相互作用统一起来。
% 而极高能标的理论本身也是当代物理学的核心问题之一。2012年,希格斯粒子(Higgs)的实验探测补齐了粒子物理标准模型的最后一块拼图,
相关方面理论建构的尝试颇多,但缺乏实验的检验,这主要是因为检验更高能标理论所需的实验条件极其苛刻。
目前人造设备所能达到的最高能标基于欧洲大型强子对撞机(Large Hadron Collider, LHC),约为$10^4\,{\rm GeV}$,这与人们预期的“统一理论”之间还有巨大的鸿沟,且在临近的几个数量级内缺乏十分有价值的预言。鉴于大型对撞机的建设成本极其高昂,目前可能的资金投入并不乐观。
% 但在地球之外,存在着诸多天然的“高能物理实验室”,比如超新星爆发、活动星系核、伽马射电暴等极端天体物理过程能标可达$10^6\,{\rm GeV}$以上,它们产生的中微子和宇宙射线等可以成为探测高能标新物理的窗口。
而原初宇宙中最早的暴胀时期的能标高达$10^{13}-10^{14}\,{\rm GeV}$,%与普朗克能标$10^{19}\,{\rm GeV}$接近,
是检验“统一理论”的天然实验室。
% 这一时期可能存在哪些物理过程完全在人类现有知识体系之外,但却与“统一理论”的构建和预言相关。
因此,两个角度殊途同归,将我们导向“世界本源”问题的一种现代化表述:“原初宇宙存在哪些高能物理过程”,这不仅是现代宇宙学的主要目标之一,也将为高能物理研究提供丰富的现象。

\section{暴胀理论及其观测证据}
% 以“新物理”的标准来看,电弱相变之前都可以称为物理学理论无法解释的“原初宇宙”,包括暴胀、reheating(热大爆炸)、大统一相变等。本文专注于暴胀部分。
尽管从寻找新的物理现象的角度来说,电弱相变以前的物理过程都有相当的探索价值,我们的研究将聚焦于其中时间最早、能标最高的暴胀时期。
暴胀理论(Cosmic Inflation)\cite{guth1981inflation}缓和了大爆炸宇宙学的奇点问题,将宇宙的起源描述为一段指数级膨胀过程,在短至$10^{-34}$秒的时间内宇宙膨胀了约$e^{60}$倍,原本极小的一片区域演化为今天的可观测宇宙,因此今天的宇宙在空间上十分平坦,且在大爆炸宇宙学中没有因果联系的两点可以通过对暴胀的回溯建立因果联系,由此解释了CMB观测中的“平坦性疑难”和“视界疑难”。

暴胀同时给宇宙中物质密度的涨落给出了自然的起源:暴胀期间共动坐标下的视界 (Comoving Horizon) 逐渐变小,原本视界内的量子涨落按照共动尺度从大到小依次出视界,被固定为经典的“原初涨落”(这个过程也称为 freeze-in),在暴胀结束后重新进入视界,之后在引力的作用下演化,最终形成我们观测到的大尺度结构。
因此通过对大尺度结构的观测,我们可以反推原初涨落的统计性质,进而了解暴胀时期的物理过程。
事实上,因为CMB之前的宇宙对光子不透明,而中微子、引力波等新信使尚处于发展初期,原初宇宙目前只能通过CMB和LSS进行间接观测。
暴胀理论现有的观测证据也全部来自对物质密度涨落统计性质的预言。%零阶疑难作为motivation,一阶涨落验证,高阶涨落发掘。
其一是原初涨落进入视界后的演化预言了重子声波振荡(Baryon Acoustic Oscillations, BAO)的存在,在CMB和LSS中都得到了验证。
其二是暴胀期间哈勃参数的变化很小,而原初涨落的幅度和其出视界时刻的哈勃参数成正比,因此不同尺度的原初功率谱应近似相同(不是严格相同,因为哈勃参数仍有微小的演化),参数化的表示是谱指数$n_s$接近于1。普朗克卫星2018年的数据给出了$n_s=0.965\pm0.004$\cite{planck18},在CMB可观测的尺度范围内证实了原初功率谱满足近似尺度不变的预言。

\section{原初非高斯性} %/宇宙学对撞机

% 最后,简单的慢滚暴胀理论预言原初涨落近似为高斯场,度量非高斯性大小的参数$f_{\rm NL}$受到慢滚参数的压低$f_{\rm NL} \sim \mathcal{O}(\epsilon)$。CMB的观测暂时也没有测到明显的非高斯性,且给出了一定的限制,比如普朗克卫星2018年的数据给出了目前最严格的限制$f_{\rm NL}=-0.9\pm5.1$\cite{planck2020png}。
对暴胀时期的观测目前主要局限于两点统计(如功率谱),而粒子物理告诉我们,两点统计描述的是暴胀期间粒子的自由传播(free propagation),粒子间的相互作用则需要更高阶的统计学量,其中蕴含着丰富的物理现象,是探索极高能标下物理理论的关键\cite{meerburg2019PNG}。
高阶统计描述的是系统相对于高斯随机场的偏离,在暴胀结束时的原初势场中,这种偏离被称为原初非高斯性(Primordial Non-Gaussianity, PNG)。
描述非高斯性的最低阶的统计学量是三点统计,如双谱(bispectrum)和三点相关函数(three-point correlation function, 3PCF),由于计算难度随统计的阶数增长迅速,现在的研究多聚焦于两点和三点统计。

% 可能存在多种场 暴胀子(inflaton),曲率子(curvaton)
从暴胀的物理来看,我们目前仅知应存在一种驱动暴胀的场,称为“暴胀子”(inflaton),对于暴胀子的性质以及是否存在其它的场则知之甚少。
% 比如生成原初涨落的机制可能是暴胀子本身的量子涨落,也可能是其它场或暴胀子自身额外的自由度,总之可以唯象地将这个自由度称为“曲率子”(curvaton, 记为$\sigma$)。
% 早期的研究从暴胀子的势能、动能、及多场暴胀的角度
% inflaton的自相互作用可以看作有额外的粒子传递
% 高能理论告诉我们暴胀期间大概率存在额外的场(很多文献这么写)
根据暴胀期间可能存在的额外粒子$\sigma$的质量可以分为三种情况:
其一是$\sigma$的质量远大于暴胀期间的哈勃参数($m_\sigma \gg H$),因此它一旦产生就会迅速衰变,这种情况等效于单场暴胀;
其二是$\sigma$接近于零质量粒子($m_\sigma \ll H$),可以在衰变前传播一段时间,就是通常的多场暴胀;
最后一种情况介于二者之间,$m_\sigma \sim H$,被称为 quasi-single field\cite{chen2010quasingle},是宇宙对撞机(Cosmological Collider)的主要研究内容\cite{arkani-hamed2015coco}。
% 它们预言的双谱各不相同。%对三角形形状的依赖性
它们预言的原初双谱是区分这三种情况,甚至进一步限制$\sigma$的质量、自旋等性质的主要途径。

原初曲率涨落(curvature fluctuation)的双谱可以参数化为
% \begin{equation}
%     B_{\zeta}(k_1,k_2,k_3) \sim \frac{\Pcal_{\zeta}^2}{{\qty(k_1 k_2 k_3)}^2}  S(k_1,k_2,k_3),
% \end{equation}
% 其中$\sim$表示根据功率谱和双谱的定义可能差一些形如$(2\pi)^n$的常数因子,
\begin{equation}
    B_{\zeta}(k_1,k_2,k_3) = \frac{{(2\pi)}^4 \Pcal_{\zeta}^2}{{\qty(k_1 k_2 k_3)}^2}  S(k_1,k_2,k_3)\,,
\end{equation}
其中
约化后的原初功率谱近似尺度无关$\Pcal_{\zeta}\sim 2\times 10^{-9}$,原初双谱对尺度大小的依赖主要在因子${\qty(k_1 k_2 k_3)}^{-2}$里,而描述三角形形状依赖的因子$S(k_1,k_2,k_3)$被称为形状函数(shape function),包含着双谱中的绝大部分信息。
% $\Pcal_{\zeta}=\frac{H^2}{8\pi^2\epsilon M_{\rm Pl}}\sim 2\times 10^{-9}$
暴胀模型对形状函数的预言通常很复杂,为了进行观测限制,人们构建了一些近似的简化样板(template),%常见的样板有三种:局域型(local, LPNG),等边型(equilateral, EPNG)和正交型(orthogonal)。
其中“局域型”(local, LPNG)% $S_{\rm local}\left(k_1, k_2, k_3\right)\sim f_{\rm NL}^{\rm loc} \qty(\frac{k_3^2}{k_1 k_2}+2 {\rm ~cyc. })$ 
是区分单场和多场暴胀模型的关键,且在星系功率谱上的信号最强,得到了最多研究\cite{dalal2008bk,ross2013bosspng,putter2017bk,karagiannis2018png,ferraro2019spec,planck2020png,cagliari2023ebosspng,chaussidon2024desipng}。
目前CMB对LPNG的测量最为精确,普朗克卫星2018年的数据给出LPNG的幅度为$f_{\rm NL}^{\rm loc}=-0.9\pm 5.1$\cite{planck2020png};LSS的测量也紧随其后,最新的DESI第一年(Y1)数据给出$1~\sigma$的误差约为$10$左右\cite{chaussidon2024desipng}。
鉴于CMB观测已接近宇宙方差\cite{babich2004pbscmb,sefusatti2007pngb2},可以预期在未来五到十年,LSS的测量精度会接近并最终超过CMB\cite{achucarro2022inflation,zhao2024must}。
% 接近正三角形的构型(Equilateral PNG)$f_{\rm NL}^{\rm equil}=-26\pm 47$\cite{planck2020png}告诉我们暴胀子的耦合强度不大;
% constraint on equil PNG tells us "Inflatons are weakly coupled".
% 而(Orthogonal PNG) $f_{\rm NL}^{\rm orth}=-38\pm 24$\cite{planck2020png}则。
另外随着理论工具的发展,物理学家们也提出了一些更具物理含义的non-local template,并开始在观测数据中搜寻相应的信号\cite{sohn2024CCCMB,cabass2024boss,green2024boss}。
% 唯象
% 随着理论计算工具的发展,物理学家们逐步建立起了原初非高斯性双谱的完整理论,更进一步,Sohn等人\cite{sohn2024CCCMB}通过对CMB数据的分析第一次完备地搜寻了这些双谱信号。
% 我们致力于在LSS中做同样的完备搜寻。

相比CMB而言,LSS的观测数据来自宇宙相对晚期,引力的非线性演化、宇宙学探针的偏袒(bias)等效应都会影响到对PNG信号的搜寻,从PNG的理论预言到最终的观测量之间的建模比CMB双谱更为复杂。
不过随着BOSS、DESI等光谱巡天项目的开展,以及下一代光谱巡天和中性氢巡天等项目的启动,LSS的观测数据处理方法在快速发展,包括利用多探针(multi-tracer)方法提升测量精度\cite{barreira2023mtpng,sullivan2023mtpng}和含有PNG的模拟\cite{coulton2022quijotepng,adame2024unitpng,hadzhiyska2024abacuspng}等。
% 不过目前这类研究主要集中在LPNG的测量上,对于non-local PNG,虽然早有相应的算法\cite{scoccimarro2012nonlocalpng,wagner2010pshmf,regan2012uningic},但相应的模拟\cite{coulton2022quijotepng,goldstein2024quijotecc}仍比较少。
PNG测量正逐渐成为现代宇宙学观测的核心科学目标之一,揭开宇宙起源时分之高能物理过程的神秘面纱,进一步回答“世界本源”问题的条件正日趋成熟。


% \section{宇宙学观测进展}

% 大尺度巡天的进展、预言

\chapter{研究内容}
% 、工作特色及难点、预期成果及可能的创新点
% 课题研究内容,研究方案,工作特色及难点,预期成果和可能的创新点以及论文工作的总体安排等
基于PNG理论和大尺度结构观测快速发展的背景,我们计划开发一系列有助于通过宇宙大尺度结构测量PNG的宇宙学观测数据分析方法。

大尺度结构宇宙学有一套成熟的数据分析方法,以贝叶斯推断为核心,% 以计算非线性引力演化的N体数值模拟为核心
其相对于CMB数据分析的一个主要区别在于建模和数据协方差(covariance)估计中十分依赖N体数值模拟求解暗物质粒子的非线性引力演化。
基于演化后的暗物质粒子分布定义暗物质晕(halo),再通过观测的成团性(clustering)进行从暗物质晕到探针样本的模型拟合,最终得到模拟的探针样本,用于统计学量的非线性建模或校准。
协方差则通过运行多组不同初始条件(Initial Condition, IC)的模拟,计算它们的统计误差来估计。
需要强调的是,在标准的分析中,各组模拟IC的差别仅在于随机数种子不同,其统计性质是一致的——都是高斯随机场。
% 这一过程不仅在计算观测统计量的样本协方差(covariance)时不可或缺,对于观测统计量的建模也至关重要。
而在PNG的测量中,当数据质量足够好时,为了数据分析过程中的一致性,我们需要含有对应PNG的数值模拟。基于定义,含PNG的模拟与标准宇宙学分析中用到的模拟差别就在于在生成N体数值模拟的初始条件时,在引力势场中加入PNG的成分。

\section{局域型原初非高斯性}

对于LPNG,修改引力势场的方式比较简单。随着近年来相应的研究逐渐增多,数组大型模拟项目将LPNG纳入参数空间\cite{coulton2022quijotepng,adame2024unitpng,hadzhiyska2024abacuspng},一些快速N体模拟程序也内嵌了加入LPNG的功能\cite{feng2016fastpm,enriquez2022picola}。
我们将在目前的研究基础上进一步探索多探针联合测量方法、高阶统计和非标准统计。这些方法将在模拟数据上得到充分验证后,再被用于实际的观测数据。

目前DESI的测量\cite{chaussidon2024desipng}只使用了亮红星系(Luminous Red Galaxy, LRG)和类星体(Quasar, QSO)两种探针的功率谱,这两类探针的红移重合度很小,所以对多探针的应用仅限于在最后一步的贝叶斯分析中,将两个样本给出的后验分布(posterior)合并,而不考虑不同探针之间的互相关统计。与LRG红移重合较高的发射线星系(Emission Line Galaxy, ELG)因为系统误差较为严重,将其纳入分析是否有助于结果的提升仍在研究之中。
高阶的三点统计也因统计误差尚在研究中,而暂时没有被纳入PNG的测量中。
虽然在这两方面相对谨慎,DESI Y1的数据分析中却引入了一种新的样本加权方法,像这样的新的数据分析方法,以及宇宙学中新开发出的多种非标准统计(如空洞(void)统计、小波变换(wavelet transform)、拓扑量等),都需要在模拟数据上进行充分的验证。

此外,随着样本量的积累,也可以人为地划分子样本来进行“多探针”的数据分析。初步的研究表明,两组样本的bias差别越大,LPNG的测量精度越高\cite{barreira2023mtpng},基于此类结论采用巧妙的方法划分星系子样本或有助于提高LPNG的测量精度。目前此类研究多利用模拟数据的暗物质晕样本(halo)进行分析,我们将在DESI合作组中基于第三年(Y3)的数据通过晕占据数模型(Halo Occupation Distribution, HOD)等方法,生成模拟的星系样本,在跟接近实际观测样本的层面进行多探针分析,以测试多探针联合分析在实际测量中的有效性。
最后值得一提的是,从 DESI II 开始,高红移的莱曼截断星系(Lyman Break Galaxy, LBG)逐渐进入光谱巡天的视野,到以MUST为代表的第五代巡天项目中,LBG将成为主要的探针之一。这将为LSS巡天带来全新的窗口,特别有利于PNG的测量。同样作为新窗口到来的还有以SKA为代表中性氢巡天项目,我们也将关注中性氢与星系的多探针联合观测。

\section{宇宙对撞机}

鉴于LPNG的测量在快速发展中,我们很有可能在不久的将来测量到原初非高斯性信号。虽然LPNG能限制一大批理论模型,但对了解暴胀时期的物理更有启发性的是另一类与粒子物理模型联系更为紧密的template,其中最具代表性的来自宇宙对撞机(Cosmological Collider, CoCo)的预言。
% 在传统的LPNG之外,我们注意到一种新的,更具物理含义的PNG template,这种template由宇宙对撞机预言——
假设暴胀期间存在额外的粒子$\sigma$,它将通过与inflaton的耦合贡献PNG,其信号双谱在压缩态极限(squeezed-limit)下的近似具有非常独特的形式——在 $\ln k$ 空间的余弦振荡,即“宇宙对撞机信号”形如
\begin{equation}
    \lim_{k_3\ll k_1\approx k_2} S(k_1,k_2,k_3)|_{\rm signal} \propto f_{\rm NL}^{\rm cc} \left(\frac{k_3}{k_1}\right)^\alpha \cos\left[\omega \ln\left(\frac{k_3}{k_1}\right) + \varphi\right]\,, \label{eq:coco_signal}
\end{equation}
其中有四个未知参数:$f_{\rm NL}^{\rm cc}$是信号的大小,$\alpha$衡量信号衰减的速度,振荡频率$\omega=\sqrt{m^2/H^2-9/4}$与$\sigma$粒子的质量直接相关,$\varphi$是振荡的相位。

为了最大限度利用LSS数据搜寻CoCo信号,我们需要针对式~(\ref{eq:coco_signal})中的template开发相应的数据分析方法。
% 不同于LPNG,这种振荡信号在较小尺度依然存在,
% 作为第一步,
在我的博士学位论文工作中,我们将先准备相应的模拟,并在暗晕功率谱上做简单的测试,为以后探索更有针对性的数据分析方法打下基础。

% 式~(\ref{eq:coco_signal})中有四个未知参数:
% \begin{itemize}
%     \item $f_{\rm NL}^{\rm cc}$是信号的大小,受到$f_{\rm NL}^{\rm loc}$的限制,作为初步的数值模拟研究,我们的工作重点在于在数值模拟中还原理论预言的双谱,并基于模拟开发数据分析方法,因此为避免数值计算误差的影响,我们将适当放大该参数的取值;
%     \item $\alpha$衡量信号衰减的速度,简单的数图给出的预言值是$1/2$,我们的工作将采用这一特征值;
%     \item 振荡频率$\omega=\sqrt{m^2/H^2-9/4}$,与$\sigma$粒子的质量直接相关,是重要的观测量,根据理论预言,我们将在$1$的数量级上对该参数进行探索;
%     \item 振荡相位$\varphi$,为简单起见,暂时设为$0$.
% \end{itemize}

目前有关CoCo信号的模拟仅见Quijote一例\cite{goldstein2024quijotecc},但其为了保证计算速度使用了进一步近似的template,无法还原余弦振荡的形式。
我们注意到有另一种较慢的算法可以处理任意 template \cite{wagner2010pshmf,regan2012uningic},只是过去探测 non-local PNG 的时机不成熟,获得的关注较少。我们计划基于这种通用算法生成含有CoCo信号的IC,并从数值模拟的角度测试这一算法的有效性。
我们将第一次将有振荡形式的shape function引入N体数值模拟,工作重点在于在数值模拟中正确还原理论预言的双谱,并基于模拟开发和测试数据分析方法,因此我们暂不追求还原宇宙中真实的双谱,而是将在理论的指导下适当探索参数空间。
具体来说,
\begin{itemize}
    \item $f_{\rm NL}^{\rm cc}$受到$f_{\rm NL}^{\rm loc}$的限制,因此不会很大,但我们为避免数值计算误差的影响,将适当放大该参数的取值;
    \item 简单的数图给出的预言值是$\alpha=1/2$,我们的工作将采用这一特征值;
    \item $\omega$的预言集中在几到几十,我们将在$1$的数量级上对该参数进行探索;
    \item 为简单起见,振荡相位$\varphi$暂设为$0$.
\end{itemize}

此外宇宙对撞机的预言中除了“信号”还有“背景”\cite{arkani-hamed2015coco},这种背景在等边三角形中取极大值,因此可以用“等边型”(equilateral, EPNG)描述。现有含EPNG的模拟使用的算法和Quijote对CoCo信号的处理一脉相承,也存在对理论template的进一步近似\cite{scoccimarro2012nonlocalpng}。我们使用的通用算法可以很容易地将宇宙对撞机信号替换为EPNG信号,因此在生成IC这一步,我们还将检验EPNG现有近似的准确度。

有了含有特定PNG的IC,我们就可以通过N体数值模拟程序计算引力演化,从而得到暗物质晕模拟样本。
我们将依前所述探索式~(\ref{eq:coco_signal})的参数空间,作为初步的探索,我们将主要尝试$\omega$的不同取值,通过简单的费舍尔预测(Fisher forecast),我们将初步了解大尺度结构对宇宙对撞机信号的限制能力。
进一步地,我们将通过数值模拟结果对暗晕功率谱进行理论建模,初步推测对CoCo信号的建模可以通过修改LPNG的模型迁移而来,我们将在模拟结果上测试这一点。
完成建模后,就可以构建完整的数据分析方法,并在模拟结果上测试。
通过测试的数据分析方法可以直接迁移到真实的观测数据上,用于CoCo信号的搜寻。
同时,通过对模拟数据(或观测数据)协方差的简单外推,我们可以进一步了解下一代LSS巡天数据对宇宙对撞机信号的限制能力。

% 在这条分析和预测宇宙对撞机信号的主线之外,我们还可以对EPNG的背景做相应的分析,为信号与背景的协同搜寻打下基础。
在CoCo信号的两点统计分析方法完成后,我们还可以尝试加入三点统计协同分析。CoCo的EPNG背景在三点统计上更明显,因此两点和三点的协同分析将适用于CoCo信号和背景的同步搜寻,这将拓宽同等数据质量下可以限制的参数空间。不过由于时间限制,这部分工作可能不会包含在我的博士论文中,而是留待未来的工作。

总之,我们将一方面参与推动LPNG的测量,另一方面将更有潜力的宇宙对撞机信号搜寻引入大尺度结构宇宙学的视野,为即将到来的LSS巡天数据准备分析方法,推动现代宇宙学观测探索原初宇宙中的物理现象,最终接过回答“世界本源”问题的接力棒。
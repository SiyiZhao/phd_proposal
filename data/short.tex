% !TeX root = ../proposal.tex

% \section{释题:从“世界的本源”到“原初宇宙的高能物理”}

% 现代科学的起源可以追溯到古希腊哲学,在后者为前者提供的诸多重要问题中,最早可考的一个是“世界的本源是什么”。
翻开任意一本哲学史,开篇通常记录着人类文明对于“智慧”的求索起源于一个著名的问题:“世界的本源是什么”。哲学史本身并未持续给出与时俱进的答案,而是将这一任务交给科学。
% 作为书写人类思想之发轫的哲学史开篇往往是古希腊的一个著名问题:“世界的本源是什么”,
时至今日,现代版本的“原子论”及其进一步发展后的“粒子物理标准模型”,从组份细分的角度来说,就是目前对这个问题最好的答案。
但相应的物理理论本身提示我们这并非终点——“万有引力”尚不在“标准模型”中,将时空纳入这个框架仍具有理论上的困难。
% 更不用说天文学还提出了“暗物质”“暗能量”这些几乎完全未知的能量组份。
% 近五百年现代物理学的发展让我们理解了日常生活尺度上种种现象的基本规律,虽然在这个尺度上,复杂系统仍有众多悬而未决的问题和转换为生产力的无限潜力,但看起来与“世界的本源”问题更接近的是与日常生活相去甚远的极小与极大两个尺度,即“衔尾蛇”的头与尾。

从时间的角度切入这个问题则涉及到宇宙观的演变。
直到1920年,沙普利(Shapley)和柯蒂斯(Curtis)还在“大辩论”中争论银河系是不是宇宙的全部,而很快哈勃(Hubble)的观测就将人类的宇宙观从银河系扩大到包含诸多星系的广阔空间。%将康德的/哲学中“岛宇宙”引入/转译为科学中的星系。
随后的观测中提出的哈勃定律则进一步启示了宇宙在膨胀,将宇宙观从简单假设的“静态”引向有实证支持的“动态”;%从简单的静态假设引向有实证的动态断言;
% 一百多年前,哈勃通过观测遥远星系的红移发现宇宙在膨胀,将宇宙观从简单的静态假设引向有实证的动态;且对膨胀状态在时间上做反方向的外推,
且基于膨胀宇宙论做合理的外推,宇宙在时间上的“起源”也成为一个自然的概念。
% 外推得到的宇宙早期应处于高温高密的状态,后来的微波背景辐射(Cosmic Microwave Background, CMB, 1964)、原初核合成(Big Bang Nucleosynthesis, BBN, 1960s)等观测佐证了这一点,九十年代末基于更精确的超新星测距得出“宇宙在加速膨胀”更是将膨胀的宇宙论确立为现代宇宙学的标准模型。
之后在微波背景辐射(Cosmic Microwave Background, CMB, 1964)、原初核合成(Big Bang Nucleosynthesis, BBN, 1960s)、宇宙大尺度结构(Large Scale Structure, LSS, 1970s)等观测证据的支持下,“大爆炸宇宙学”逐渐成为现代科学公认的宇宙观,
到了九十年代末基于超新星测距发现“宇宙在加速膨胀”,进而引入暗能量的概念后,现代宇宙学的标准模型已经能至少在零阶近似下准确地描述宇宙的绝大部分历史。
仍然未知的那部分就是电弱相变之前的“原初宇宙”,% 包括暴胀、reheating(热大爆炸)、大统一相变等。本文专注于暴胀部分。
或者说宇宙的“起源”。因为这一时期的能标极高,并不在现代物理学知识体系之内。
% 却也把“宇宙诞生之初”这个“关乎起源”的概念置于现代物理学知识体系之外——这段特殊的历史时期应该处于极高的能标,可以用简洁优美的统一理论描述,但我们在探索这个“统一理论”的过程中却遇到了极大的困难。

理论物理学家大都认为,目前经过实验认证的最深刻的两大理论——广义相对论(目前所知描述引力的最佳理论)和量子场论(描述其它三种相互作用的有力工具)只是某种更高能标的“统一理论”的低能近似,后者应能同时自然地描述全部四种相互作用。
% 更高能标理论的低能近似,而更高能标的理论应能将四种相互作用统一起来。
% 而极高能标的理论本身也是当代物理学的核心问题之一。2012年,希格斯粒子(Higgs)的实验探测补齐了粒子物理标准模型的最后一块拼图,
相关理论建构的尝试很多,但缺乏实验的检验,这主要是因为检验更高能标理论所需的实验条件极其苛刻。
目前人造设备所能达到的最高能标基于欧洲大型强子对撞机(Large Hadron Collider, LHC),约为$10^4\,{\rm GeV}$,这与人们预期的“统一理论”之间还有巨大的鸿沟,且在临近的几个数量级内缺乏十分有价值的预言。鉴于大型对撞机的建设成本极其高昂,目前人们对其可能的资金投入并不乐观。
% 但在地球之外,存在着诸多天然的“高能物理实验室”,比如超新星爆发、活动星系核、伽马射电暴等极端天体物理过程能标可达$10^6\,{\rm GeV}$以上,它们产生的中微子和宇宙射线等可以成为探测高能标新物理的窗口。
而“原初宇宙”中最早的“暴胀时期”的能标高达$10^{13}-10^{14}\,{\rm GeV}$,%与普朗克能标$10^{19}\,{\rm GeV}$接近,
是检验“统一理论”的天然实验室。
% 这一时期可能存在哪些物理过程完全在人类现有知识体系之外,但却与“统一理论”的构建和预言相关。
因此,两个角度殊途同归,将我们导向“世界本源”问题的一种现代表述:“原初宇宙存在哪些高能物理过程”,这不仅是现代宇宙学的主要目标之一,也将为高能物理研究提供丰富的现象。

% \chapter{研究背景/文献综述}

% \section{暴胀}
% 以“新物理”的标准来看,电弱相变之前都可以称为物理学理论无法解释的“原初宇宙”,包括暴胀、reheating(热大爆炸)、大统一相变等。本文专注于暴胀部分。
虽然从寻找新的物理现象的角度来说,电弱相变以前的物理过程都有相当的探索意义,我们的研究将集中在其中最早、能标最高的暴胀时期。
暴胀理论(Cosmic Inflation)\cite{guth1981inflation}缓和了大爆炸宇宙学的奇点问题,将宇宙的起源描述为一段指数级膨胀过程,在短至$10^{-34}$秒的时间内宇宙膨胀了约$e^{60}$倍,原本极小的一片区域演化为今天的可观测宇宙,因此今天的宇宙在空间上十分平坦,且在大爆炸宇宙学中没有因果联系的两点可以通过对暴胀的回溯建立因果联系,由此解释了CMB观测中的“平坦性疑难”和“视界疑难”。

暴胀同时给宇宙中物质密度的涨落给出了自然的起源:暴胀期间共动坐标下的视界 (Comoving Horizon) 逐渐变小,原本视界内的量子涨落按照共动尺度从大到小依次出视界,被固定为经典的原初涨落(这个过程也称为 freeze-in),在暴胀结束后重新进入视界,之后在引力的作用下演化,最终形成我们观测到的大尺度结构。
因此我们可以通过对LSS的观测来反推暴胀结束时的原初场的统计性质,进而推断暴胀时期的物理过程。
事实上,暴胀理论目前的观测支持全部来源于其对物质密度涨落的统计性质的预言。
其一是原初涨落进入视界后的演化预言了重子声波振荡(Baryon Acoustic Oscillations, BAO)的存在,在CMB和LSS中都得到了验证。
其二是暴胀期间哈勃参数的变化很小,而原初涨落的幅度和其出视界时刻的哈勃参数成正比,因此不同尺度的涨落在出视界时刻的幅度也应是近似相同的(不是严格相同,因为暴胀期间哈勃参数有微小的演化),参数化的表示是功率谱的谱指数$n_s$接近于1。普朗克卫星2018年的数据给出了$n_s=0.965\pm0.004$\cite{planck18},且在可探测的尺度范围内都满足近似尺度不变的预言。

% \section{原初非高斯性/宇宙学对撞机}
% 最后,简单的慢滚暴胀理论预言原初涨落近似为高斯场,度量非高斯性大小的参数$f_{\rm NL}$受到慢滚参数的压低$f_{\rm NL} \sim \mathcal{O}(\epsilon)$。CMB的观测暂时也没有测到明显的非高斯性,且给出了一定的限制,比如普朗克卫星2018年的数据给出了目前最严格的限制$f_{\rm NL}=-0.9\pm5.1$\cite{planck2020png}。
现有的测量局限于两点统计(如功率谱),粒子物理中的作用量告诉我们,两点统计描述的是暴胀期间粒子的自由传播(free propagation),而粒子间的相互作用则需要更高阶的统计学量,其中蕴含着丰富的物理现象,是探索极高能标理论的天然窗口。
高阶统计描述的是系统相对于高斯随机场的偏离,在暴胀结束时的原初势场中,这种偏离被称为原初非高斯性(Primordial Non-Gaussianity, PNG)。
描述非高斯性的最低阶的统计学量是三点统计,如双谱(bispectrum)和三点相关函数(three-point correlation function, 3PCF),由于计算难度随统计的阶数增长迅速,现在的研究多聚焦于两点和三点统计。



可能存在多种场 暴胀子(inflaton),曲率子(curvaton)
% 比如squeezed-limit/local 
% 接近正三角形的构型(Equilateral PNG)$f_{\rm NL}^{\rm equil}=-26\pm 47$\cite{planck2020png}告诉我们暴胀子的耦合强度不大;
% constraint on equil PNG tells us "Inflatons are weakly coupled".
% 而(Orthogonal PNG) $f_{\rm NL}^{\rm orth}=-38\pm 24$\cite{planck2020png}则。


% 唯象
随着理论计算工具的发展,物理学家们逐步建立起了原初非高斯性双谱的完整理论,更进一步,\cite{sohn2024CCCMB}通过对CMB数据的分析第一次完备地搜寻了这些双谱信号。
我们致力于在LSS中做同样的完备搜寻。
相比CMB而言,LSS的观测数据来自宇宙相对晚期,引力的非线性演化、宇宙学探针的偏袒(bias)等效应都会影响到对PNG信号的搜寻,从PNG的理论预言到最终的观测量之间的建模比CMB双谱更为复杂。
但另一方面,随着BOSS、DESI等光谱巡天项目的开展,星系光谱巡天的观测数据在积累,相应的数据处理方法也在快速发展之中,

% \section{宇宙学观测进展}

% \chapter{研究内容、工作特色及难点、预期成果及可能的创新点}
% !TeX root = ./proposal.tex

% 论文基本信息配置

\thusetup{
  output = electronic,
  thesis-type = proposal,
  title  = {探索原初高能物理的\\现代宇宙学观测方法}, %{现代宇宙学观测探索原初高能物理}, % {叩宇穷宙:现代宇宙学观测\\探索原初高能物理},
  title* = {To Detect Primordial Physics with Late-time Cosmology Observations},
  degree-category  = {理学博士},
  degree-category* = {Doctor of Philosophy},
  department = {天文系},
  discipline  = {天文学},
  discipline* = {Astronomy},
  author  = {赵思逸},
  author* = {Zhao Siyi},
  %
  % 学号
  % 仅当书写开题报告时需要(同时设置 `thesis-type = proposal')
  %
  student-id = {2021312349},
  %
  % 指导教师
  %   中文姓名和职称之间以英文逗号“,”分开,下同
  %
  supervisor  = {赵成, 教授},
  supervisor* = {Professor Zhao Cheng},
  %
  % 是否在中文封面后的空白页生成书脊(默认 false)
  %
  include-spine = false,
}

% 载入所需的宏包

% 定理类环境宏包
\usepackage{amsthm}

\thusetup{
  %
  % 数学字体
  % math-style = GB,  % GB | ISO | TeX
  math-font  = xits,  % stix | xits | libertinus
}

% 物理学符号宏包
\usepackage{physics}

% 表格加脚注
\usepackage{threeparttable}

% 表格中支持跨行
\usepackage{multirow}

% 固定宽度的表格。
% \usepackage{tabularx}

% 跨页表格
\usepackage{longtable}

% 算法
\usepackage{algorithm}
\usepackage{algorithmic}

% 量和单位
\usepackage{siunitx}
% 自定义command
\newcommand{\Pcal}{\mathcal{P}}
\newcommand{\PG}{^{\rm G}} % primordial Gaussian case
\newcommand{\LPNG}{^{\rm local}} % primordial non-Gaussian case
\newcommand{\ks}{k_{\rm s}} % short wavelength
\newcommand{\kl}{k_{\rm l}} % long wavelength

% 参考文献使用 BibTeX + natbib 宏包
% 顺序编码制
\usepackage[sort]{natbib}
\bibliographystyle{thuthesis-numeric}

% 定义所有的图片文件在 figures 子目录下
\graphicspath{{figures/}}

% 数学命令
\makeatletter
\newcommand\dif{%  % 微分符号
  \mathop{}\!%
  \ifthu@math@style@TeX
    d%
  \else
    \mathrm{d}%
  \fi
}
\makeatother

% hyperref 宏包在最后调用
\usepackage{hyperref}
